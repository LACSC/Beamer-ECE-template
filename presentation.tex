\documentclass{beamer}
\usepackage[latin1]{inputenc}
\usepackage{amsfonts}					% special fonts support (math, etc)
\usepackage{color}
\usepackage{graphicx}
\usepackage{listings}
\usepackage{pdftricks}
\usepackage{hyperref}					% allow interal links
\usepackage{makeidx}
\begin{psinputs}
  \usepackage{pstricks}
  \usepackage{pstricks-add}
  \usepackage{multido}
  \usepackage{epsfig}
  \usepackage{pst-grad} % For gradients
  \usepackage{pst-plot} % For axes
  \usepackage{pst-tree} % For trees
  \usepackage{pst-3d} % For three dimensionnal graph
\end{psinputs}


\usetheme{Phil}
\useinnertheme[shadow=true]{rounded}

% titlepage. This configuration is used in the first frame
\title[Linux kernel course]{SLIDES TITLE}
\author{Philippe Thierry (philippe.thierry@reseau-libre.net)}
\institute{�cole Centrale d'�lectronique}
\date{\date}

% create a "frameheader" command. This command just show the ECE & Thales
% logos on the feame
\newcommand{\frameheader}{%
\vspace{-7mm}%
\begin{tabular}{p{0.5\paperwidth}p{0.5\paperwidth}}%
\includegraphics[width=0.3\paperwidth]{img/ece_header.png} & \\%
\end{tabular}%
}

% New frame title command. The title is pushed on the right and made
% italic. the double ~~ avoid the title to touch the right border of the frame
\newcommand{\ftitle}[1]{
\vspace{-7mm}
\begin{flushright}{\it #1~~}\end{flushright}
}

%% Configure lstlisting colors for C code
\definecolor{Gray}{rgb}{0.95,0.95,0.95}
\definecolor{purple}{rgb}{0.8,0,0.8}

% Configuration of the lstlisting system.
% Add some kernel specific keywords to the C language highlighting
\lstset{
% language
language=C,
classoffset=0,
morekeywords={int,void,long,char,struct,union,enum,unsigned},
keywordstyle={\color{blue}},
classoffset=1,
morekeywords={\_\_user,\_\_init},
keywordstyle={\color{red}},
classoffset=2,
morekeywords={\#define,\#endif,\#if,\#else,\#ifndef},
keywordstyle={\color{purple}},
classoffset=3,
morekeywords={\verb+@+param,\verb+@+return,brief,arg},
keywordstyle={\color{green}},
classoffset=0
%morecomment=*[n][\itshape]{/*}{*/},
% generals
basicstyle=\tiny,
numbers=left,
numberstyle=\tiny,
stepnumber=1,
numbersep=5pt,
backgroundcolor=\color{Gray},
showtabs=false,
frame=single,
breaklines=true,
captionpos=b
}

% encapsulating alt command in order to avoid to write to times the text.
% first arg is the slide on which the red color is used, the second arg is
% the text to manage
\newcommand{\myalt}[2]{\alt<#1>{\color{red} #2}{\color{gray} #2}}

% support for index
\makeindex
\newenvironment{theindex}
 {\let\item\par
  %definitions for subitem etc
  }{}
\newcommand\indexspace{}
% let's start...
\begin{document}

% first add frames background, using the ECE standard canvas
\setbeamertemplate{background canvas}{\includegraphics[width=\paperwidth]{img/ece.png}}

% some colors
%\definecolor{preproc}{rgb}{}

%%%%%%%%%%%%%%%%%%%%%%%%%%%%% Frame %%%%%%%%%%%%%%%%%%%%%%%%%%%%%%%%%
\begin{frame}
\frameheader
\titlepage
\end{frame}

% disactive title color. Frame titles will have no bg color
\setbeamercolor{titlelike}{fg=titleTextColor,bg=white}

%\setbeamertemplate{background canvas}[vertical shading][top=blue!80,midpoint=0.8,bottom=white]
\section{Introduction}

%%%%%%%%%%%%%%%%%%%%%%%%%%%%% Frame %%%%%%%%%%%%%%%%%%%%%%%%%%%%%%%%%
\subsection{Summary}
\frame[shrink]{
\tableofcontents
}


%%%%%%%%%%%%%%%%%%%%%%%%%%%%% Frame %%%%%%%%%%%%%%%%%%%%%%%%%%%%%%%%%
\subsection{Aim of this document}
\begin{frame}
\begin{itemize}
\item \FIXME{first aim}
\end{itemize}

\end{frame}

\section{First section}
\subsection{First subsection}

%%%%%%%%%%%%%%%%%%%%%%%%%%%%% Frame %%%%%%%%%%%%%%%%%%%%%%%%%%%%%%%%%
\begin{frame}
\end{frame}

\subsection{Second subsection}
%%%%%%%%%%%%%%%%%%%%%%%%%%%%% Frame %%%%%%%%%%%%%%%%%%%%%%%%%%%%%%%%%
\begin{frame}
\frametitle{\ftitle{frame title}}
\end{frame}


\subsection{Questions}
%%%%%%%%%%%%%%%%%%%%%%%%%%%%% Frame %%%%%%%%%%%%%%%%%%%%%%%%%%%%%%%%%
\begin{frame}
\vspace{5mm}
\begin{center}
\includegraphics[width=5cm]{img/linux_qestion}
\end{center}
\end{frame}

\subsection{Index}
%%%%%%%%%%%%%%%%%%%%%%%%%%%%% Frame %%%%%%%%%%%%%%%%%%%%%%%%%%%%%%%%%
\frame[allowframebreaks]{
\frametitle{\ftitle{Index}}
\printindex
}


\end{document}
